% Options for packages loaded elsewhere
\PassOptionsToPackage{unicode}{hyperref}
\PassOptionsToPackage{hyphens}{url}
\PassOptionsToPackage{dvipsnames,svgnames,x11names}{xcolor}
%
\documentclass[
  letterpaper,
  DIV=11,
  numbers=noendperiod,
  oneside]{scrreprt}

\usepackage{amsmath,amssymb}
\usepackage{lmodern}
\usepackage{iftex}
\ifPDFTeX
  \usepackage[T1]{fontenc}
  \usepackage[utf8]{inputenc}
  \usepackage{textcomp} % provide euro and other symbols
\else % if luatex or xetex
  \usepackage{unicode-math}
  \defaultfontfeatures{Scale=MatchLowercase}
  \defaultfontfeatures[\rmfamily]{Ligatures=TeX,Scale=1}
\fi
% Use upquote if available, for straight quotes in verbatim environments
\IfFileExists{upquote.sty}{\usepackage{upquote}}{}
\IfFileExists{microtype.sty}{% use microtype if available
  \usepackage[]{microtype}
  \UseMicrotypeSet[protrusion]{basicmath} % disable protrusion for tt fonts
}{}
\makeatletter
\@ifundefined{KOMAClassName}{% if non-KOMA class
  \IfFileExists{parskip.sty}{%
    \usepackage{parskip}
  }{% else
    \setlength{\parindent}{0pt}
    \setlength{\parskip}{6pt plus 2pt minus 1pt}}
}{% if KOMA class
  \KOMAoptions{parskip=half}}
\makeatother
\usepackage{xcolor}
\usepackage[left=1in,marginparwidth=2.0666666666667in,textwidth=4.1333333333333in,marginparsep=0.3in]{geometry}
\setlength{\emergencystretch}{3em} % prevent overfull lines
\setcounter{secnumdepth}{5}
% Make \paragraph and \subparagraph free-standing
\ifx\paragraph\undefined\else
  \let\oldparagraph\paragraph
  \renewcommand{\paragraph}[1]{\oldparagraph{#1}\mbox{}}
\fi
\ifx\subparagraph\undefined\else
  \let\oldsubparagraph\subparagraph
  \renewcommand{\subparagraph}[1]{\oldsubparagraph{#1}\mbox{}}
\fi


\providecommand{\tightlist}{%
  \setlength{\itemsep}{0pt}\setlength{\parskip}{0pt}}\usepackage{longtable,booktabs,array}
\usepackage{calc} % for calculating minipage widths
% Correct order of tables after \paragraph or \subparagraph
\usepackage{etoolbox}
\makeatletter
\patchcmd\longtable{\par}{\if@noskipsec\mbox{}\fi\par}{}{}
\makeatother
% Allow footnotes in longtable head/foot
\IfFileExists{footnotehyper.sty}{\usepackage{footnotehyper}}{\usepackage{footnote}}
\makesavenoteenv{longtable}
\usepackage{graphicx}
\makeatletter
\def\maxwidth{\ifdim\Gin@nat@width>\linewidth\linewidth\else\Gin@nat@width\fi}
\def\maxheight{\ifdim\Gin@nat@height>\textheight\textheight\else\Gin@nat@height\fi}
\makeatother
% Scale images if necessary, so that they will not overflow the page
% margins by default, and it is still possible to overwrite the defaults
% using explicit options in \includegraphics[width, height, ...]{}
\setkeys{Gin}{width=\maxwidth,height=\maxheight,keepaspectratio}
% Set default figure placement to htbp
\makeatletter
\def\fps@figure{htbp}
\makeatother
\newlength{\cslhangindent}
\setlength{\cslhangindent}{1.5em}
\newlength{\csllabelwidth}
\setlength{\csllabelwidth}{3em}
\newlength{\cslentryspacingunit} % times entry-spacing
\setlength{\cslentryspacingunit}{\parskip}
\newenvironment{CSLReferences}[2] % #1 hanging-ident, #2 entry spacing
 {% don't indent paragraphs
  \setlength{\parindent}{0pt}
  % turn on hanging indent if param 1 is 1
  \ifodd #1
  \let\oldpar\par
  \def\par{\hangindent=\cslhangindent\oldpar}
  \fi
  % set entry spacing
  \setlength{\parskip}{#2\cslentryspacingunit}
 }%
 {}
\usepackage{calc}
\newcommand{\CSLBlock}[1]{#1\hfill\break}
\newcommand{\CSLLeftMargin}[1]{\parbox[t]{\csllabelwidth}{#1}}
\newcommand{\CSLRightInline}[1]{\parbox[t]{\linewidth - \csllabelwidth}{#1}\break}
\newcommand{\CSLIndent}[1]{\hspace{\cslhangindent}#1}

\KOMAoption{captions}{tableheading}
\makeatletter
\makeatother
\makeatletter
\@ifpackageloaded{bookmark}{}{\usepackage{bookmark}}
\makeatother
\makeatletter
\@ifpackageloaded{caption}{}{\usepackage{caption}}
\AtBeginDocument{%
\ifdefined\contentsname
  \renewcommand*\contentsname{Table of contents}
\else
  \newcommand\contentsname{Table of contents}
\fi
\ifdefined\listfigurename
  \renewcommand*\listfigurename{List of Figures}
\else
  \newcommand\listfigurename{List of Figures}
\fi
\ifdefined\listtablename
  \renewcommand*\listtablename{List of Tables}
\else
  \newcommand\listtablename{List of Tables}
\fi
\ifdefined\figurename
  \renewcommand*\figurename{Figure}
\else
  \newcommand\figurename{Figure}
\fi
\ifdefined\tablename
  \renewcommand*\tablename{Table}
\else
  \newcommand\tablename{Table}
\fi
}
\@ifpackageloaded{float}{}{\usepackage{float}}
\floatstyle{ruled}
\@ifundefined{c@chapter}{\newfloat{codelisting}{h}{lop}}{\newfloat{codelisting}{h}{lop}[chapter]}
\floatname{codelisting}{Listing}
\newcommand*\listoflistings{\listof{codelisting}{List of Listings}}
\makeatother
\makeatletter
\@ifpackageloaded{caption}{}{\usepackage{caption}}
\@ifpackageloaded{subcaption}{}{\usepackage{subcaption}}
\makeatother
\makeatletter
\@ifpackageloaded{tcolorbox}{}{\usepackage[many]{tcolorbox}}
\makeatother
\makeatletter
\@ifundefined{shadecolor}{\definecolor{shadecolor}{rgb}{.97, .97, .97}}
\makeatother
\makeatletter
\@ifpackageloaded{sidenotes}{}{\usepackage{sidenotes}}
\@ifpackageloaded{marginnote}{}{\usepackage{marginnote}}
\makeatother
\makeatletter
\makeatother
\ifLuaTeX
  \usepackage{selnolig}  % disable illegal ligatures
\fi
\IfFileExists{bookmark.sty}{\usepackage{bookmark}}{\usepackage{hyperref}}
\IfFileExists{xurl.sty}{\usepackage{xurl}}{} % add URL line breaks if available
\urlstyle{same} % disable monospaced font for URLs
\hypersetup{
  pdftitle={Schuller},
  pdfauthor={John Waczak},
  colorlinks=true,
  linkcolor={blue},
  filecolor={Maroon},
  citecolor={Blue},
  urlcolor={Blue},
  pdfcreator={LaTeX via pandoc}}

\title{Schuller}
\author{John Waczak}
\date{7/20/2022}

\begin{document}
\maketitle
\ifdefined\Shaded\renewenvironment{Shaded}{\begin{tcolorbox}[boxrule=0pt, breakable, borderline west={3pt}{0pt}{shadecolor}, enhanced, sharp corners, frame hidden, interior hidden]}{\end{tcolorbox}}\fi

\renewcommand*\contentsname{Table of contents}
{
\hypersetup{linkcolor=}
\setcounter{tocdepth}{2}
\tableofcontents
}
\bookmarksetup{startatroot}

\hypertarget{overview}{%
\chapter*{Overview}\label{overview}}
\addcontentsline{toc}{chapter}{Overview}

A set of notes for Frederic Schuller's course on Mathematical Physics
for the Heraeus International School on Gravity and Light.

\bookmarksetup{startatroot}

\hypertarget{multilinear-algebra}{%
\chapter{Multilinear Algebra}\label{multilinear-algebra}}

Traditionally, this subsect is known as \emph{Tensor Theory}.

Multilinear algebra \emph{is the same field} as the more familiar Linear
Algebra- the only difference is we extend our mappings of consideration
from linear maps to so called \textbf{multilinear maps}. The object of
study of multilinear algebra is just \textbf{vector spaces}.

A word of warning, though: we will \emph{not} equip space(time) with a
vector space structure. If spacetime does not have anything to do with
vector spaces, why do we care? Because the smooth manifolds we will use
to model spacetime come equipped with natural linear spaces called the
\textbf{Tangent Space}, \(T_pM\)\sidenote{\footnotesize Smooth manifolds will be
  defined in Lecture 4. The concept of the Tangent Space \(T_pM\) will
  be explored in Lecture 5.}. It is here that our development of the
multilinear algebra will really come into focus.

It is beneficial to first study Vector Spaces in full (abstract) detail
before moving on for two reasons:

\begin{enumerate}
\def\labelenumi{\arabic{enumi}.}
\tightlist
\item
  To construct \(T_pM\), one needs an intermediate vector space called
  \(C^{\infty}(M**\).
\item
  Tensor techniques are most easily understood in the abstract context.
\end{enumerate}

\hypertarget{vector-spaces}{%
\section{Vector Spaces}\label{vector-spaces}}

\textbf{Definition:} An \(\mathbb{R}\) Vector Space \((V, +, *)\) is

\begin{enumerate}
\def\labelenumi{\arabic{enumi}.}
\tightlist
\item
  A set \(V\)
\item
  \(+:V\times V\to V\) (vector addition)
\item
  \(*:\mathbb{R}\times V \to V\) (scalar multiplication)
\end{enumerate}

which satisfies the following rules:

\begin{itemize}
\item
  \begin{enumerate}
  \def\labelenumi{(\Alph{enumi})}
  \setcounter{enumi}{2}
  \tightlist
  \item
    \(v + w = w + v\) for \(v,w \in V\)
  \end{enumerate}
\item
  \begin{enumerate}
  \def\labelenumi{(\Alph{enumi})}
  \tightlist
  \item
    \((u + v) + w = u + (v + w)\) for \(u,v,w \in V\)
  \end{enumerate}
\item
  \begin{enumerate}
  \def\labelenumi{(\Alph{enumi})}
  \setcounter{enumi}{13}
  \tightlist
  \item
    \(\exists 0\in V\) such that \(\forall v \in V\): \(v + 0 = v\)
  \end{enumerate}
\item
  \begin{enumerate}
  \def\labelenumi{(\Roman{enumi})}
  \tightlist
  \item
    \(\forall v \in V\), \(\exists (-v) \in V\): \(v + (-v) = 0\)
  \end{enumerate}
\item
  \begin{enumerate}
  \def\labelenumi{(\Alph{enumi})}
  \tightlist
  \item
    \(\lambda \cdot (\nu \cdot v) = (\lambda \cdot \nu) \cdot v\)
    \(\forall \lambda, \nu \in \mathbb{R}\) and \(\forall v \in V\).
  \end{enumerate}
\item
  \begin{enumerate}
  \def\labelenumi{(\Alph{enumi})}
  \setcounter{enumi}{3}
  \tightlist
  \item
    \((\lambda + \nu)v = \lambda v + \nu v\)
  \end{enumerate}
\item
  \begin{enumerate}
  \def\labelenumi{(\Alph{enumi})}
  \setcounter{enumi}{3}
  \tightlist
  \item
    \(\lambda \nu + \lambda w = \lambda(v+w)\)
  \end{enumerate}
\item
  \begin{enumerate}
  \def\labelenumi{(\Alph{enumi})}
  \setcounter{enumi}{20}
  \tightlist
  \item
    \(1\cdot v = v\)
  \end{enumerate}
\end{itemize}

Summary: Any structure of this type that satisfies these axioms is a
vector space.

Terminology: An element of a vector space is often referred to
(informally) as a \emph{vector}.\sidenote{\footnotesize You must not ask the question:
  \emph{What is a vector?}. You couldn't tell that an object is a vector
  just by looking at it. You need to check how it behaves with other
  \emph{similar} objects.}.

\hypertarget{example}{%
\subsection{Example}\label{example}}

Consider the set of polynomials \(P_n := \{p=\sum_{i=0}^{n}p_ix^i\}\)
with \(p_i\in \mathbb{R}\) and \(p:(-1,1)\to\mathbb{R}\). At this stage,
we may ask the \emph{simple} question: is \(\square:x\mapsto x^2\) a
vector? \textbf{NO!} \(\square\) is an element of the set \(P\). We have
not defined a vector space for which addition and s-multiplication make
sense.

Now, we define \(+:P\times P \to P\) whereby \((p,q) \mapsto p+q\)
defined by

\[(p+q)(x) = p(x) + q(x)\].

Similarly, we may define s-multiplication
\(\cdot: \mathbb{R}\times P \to P\) whereby

\[ (\lambda \cdot p)(x) = \lambda \cdot p(x) \]

Now that we have defined the structure, we could decide that \(\square\)
is an element of our vector space \((P, +, \cdot)\). Of course, we
should formamly prove that \((P, +, \cdot**\) does satisfy the axioms of
a vector space\ldots{}

\hypertarget{linear-maps}{%
\section{Linear Maps}\label{linear-maps}}

These are the structure-respecting maps between vector spaces:

\textbf{Definition} If \((V, +_v, \cdot_v)\) and \((W, +_w, \cdot_w)\)
are vector spaces. Then a map \(\phi:V\to W\) is called \emph{linear}
if:

\begin{enumerate}
\def\labelenumi{\arabic{enumi}.}
\tightlist
\item
  \(\phi(v+_v\tilde{v}) = \phi(v) +_w \phi(\tilde v)\)
\item
  \(\phi(\lambda \cdot_v v) = \lambda \cdot_w \phi(v)\)
\end{enumerate}

\hypertarget{example-differentiation}{%
\subsection{Example (Differentiation)}\label{example-differentiation}}

Consider \(\delta:P \to P\)\sidenote{\footnotesize Notation: we write
  \(\phi: V \xrightarrow{\sim} W\) to denote a linear map.} whereby

\[ p\mapsto \delta(p) := p'\]

defines the differentiation.

\begin{enumerate}
\def\labelenumi{\arabic{enumi}.}
\tightlist
\item
  \textbf{Linearity:}
  \[\delta(p+q) = (p+q)' = p'+q' = \delta(p) + \delta(q)\]
\item
  \textbf{Multiplication:}
  \[\delta(\lambda p) = (\lambda p)' = \lambda p' = \lambda \delta(p)\]
\end{enumerate}

\textbf{Theorem} If \(\phi:V \xrightarrow{\sim}\) and
\(\psi:W \xrightarrow{\sim} W\), then
\(\psi \circ \phi : V \xrightarrow{\sim} U\)

\hypertarget{example-1}{%
\subsection{Example}\label{example-1}}

\[\delta \circ \delta: P \xrightarrow{\sim} P\] is linear.

\hypertarget{vector-space-of-homomorphisms}{%
\section{Vector Space of
Homomorphisms}\label{vector-space-of-homomorphisms}}

Fun-fact: If \((V, +, \cdot)\), \((W, +, \cdot)\) are vector spaces,
then we can define the set

\[\text{Hom}(V,W) := \{ \phi: V \xrightarrow{\sim} W\}\]

We can make this into a vector space by

\begin{enumerate}
\def\labelenumi{\arabic{enumi}.}
\tightlist
\item
  \(+: \text{Hom}(V,W) \times \text{Hom}(V,W) \to \text{Hom}(V,W)\)
  whereby \[ (\phi, \psi) \mapsto \phi + \psi \] so that
  \[(\phi + \psi)(v) := \phi(v) + \psi(v) \]
\item
  \ldots{} and the same for s-multiplication
\end{enumerate}

\hypertarget{example-texthompp}{%
\subsection{\texorpdfstring{Example
\(\text{Hom}(P,P)\)}{Example \textbackslash text\{Hom\}(P,P)}}\label{example-texthompp}}

is a vector space. This means that \(\delta \in \text{Hom}(P,P)\),
\(\delta \circ \delta \in \text{Hom}(P, P)\), and so on until
\(\delta^{(m)} \in \text{Hom}(P,P)\)

\hypertarget{dual-vector-space}{%
\section{Dual Vector Space}\label{dual-vector-space}}

This is just a heavily used special case\ldots{}

\textbf{Definition}:
\(V^* := \{\phi:V\xrightarrow{\sim} \mathbb{R}\} = \text{Hom}(V, \mathbb{R})\)
is the set of linear maps on \(V\). We say that \((V^*, +, \cdot)\) is
the dual vector space to \(V\).

\textbf{Terminology}: We call \(\phi \in V^*\) a \emph{covector}.

\hypertarget{example-2}{%
\subsection{Example}\label{example-2}}

Consider the map \(I:P\xrightarrow{\sim}\mathbb{R}\), i.e.~\(I\in P^*\)
defined such that\sidenote{\footnotesize This is exactly how we think of the ``bras''
  working in quantum mechanics!} \begin{equation}
    I(p) := \int_0^1 dx \; p(x)
\end{equation}

\hypertarget{tensors}{%
\section{Tensors}\label{tensors}}

\textbf{Definition:} Let \((V, +, \cdot)\) a vector space. Then, an
\((r,s)\)-Tensor over \(V\) is a multilinear map \begin{equation}
    T: V^*\times \underset{r}{...} \times V^* \times V\times \underset{s}{...} \times V \xrightarrow{\sim} \mathbb{R}
\end{equation}

\hypertarget{example-3}{%
\subsection{Example}\label{example-3}}

Let \(T\) be a \((1,1)\)-tensor. Then \begin{align}
    T(\phi, + \psi, v) &= T(\phi, v) + T(\psi, v) \\ 
    T(\lambda\phi, v) &= \lambda T(\phi, v) \\ 
    T(\phi, v+w) &= T(\phi, v) + T(\phi, w) \\ 
    T(\phi, \lambda v) &= \lambda T(\phi, v)
\end{align}

What is \(T(\phi + \psi, v + w)\)? Use linearity twice: \begin{equation}
    T(\phi + \psi, v + w) = T(\phi, v) + T(\phi, w) + T(\psi, v) + T(\psi, w)
\end{equation}

\hypertarget{excursion}{%
\subsection{Excursion}\label{excursion}}

Let \(T: V^* \times V \xrightarrow{\sim} \mathbb{R}\), then we may
define the map \sidenote{\footnotesize It is a fact that for finite dimensional vector
  spaces \(V\), \((V^*)^*=V\).} \begin{align}
    \phi_T&: V \xrightarrow{\sim} (V^*)^*\\
    v&\mapsto T(\cdot, v)
\end{align}

Given a map \(\phi: V\xrightarrow{\sim} V\), we can construct the map
\begin{align}
    T_\phi &: V^*\times V \xrightarrow{\sim} \mathbb{R} \\ 
    (\varphi, v) &\mapsto \varphi(\phi(v))
\end{align}

It follows that \(T = T_{\phi_T}\) and \(\phi = \phi_{T_\phi}\)

In other words, a map from \(V\) to \(V\) contains the same data as a
map from \((V^*\times V)\) to \(\mathbb{R}\).

\hypertarget{example-4}{%
\subsection{Example}\label{example-4}}

Consider: \(g: P\times P \xrightarrow{\sim} \mathbb{R}\) such that
\begin{equation}
    (p,q) \mapsto \int_{-1}^1 dx\; p(x)q(x)
\end{equation} is a \((0,2)\)-tensor over \(P\).

In other words, inner products are tensors\ldots{}

\hypertarget{vectors-and-covectors-as-tensors}{%
\section{Vectors and Covectors as
Tensors}\label{vectors-and-covectors-as-tensors}}

\textbf{Theorem}: \begin{equation}
    \phi \in V^* \iff \phi: V\xrightarrow{\sim}\mathbb{R} \iff \phi \text{ is a } (0,1)\text{-tensor}
\end{equation}

\textbf{Theorem}: \begin{equation}
    v\in V=(V^*)^* \iff v:V^* \xrightarrow{\sim} \mathbb{R} \iff v \text{ is a } (1,0)\text{-tensor}
\end{equation}

Okay, good! We are safe using tensors for everything.

\hypertarget{bases}{%
\section{Bases}\label{bases}}

\textbf{Definition}: Let \((V, +, \cdot)\) be a vector space. A subset
\(B \subset V\) is called a basis\sidenote{\footnotesize Specifically, a \emph{Hamel}
  basis. \href{https://mathworld.wolfram.com/HamelBasis.html}{link}} if
\(\forall v\in V\), \(\exists !\) finite \(F\subset B\) such that
\(\exists ! v^1,...,v^n \in \mathbb{R}\) so that \begin{equation}
 v = v^1f_1 + ... + v^n f_n
\end{equation}

Or in words: for each vector there exists a unique subset of \(B\) such
that there exists a unique finite collection of real numbers \(v^i\) so
that \(v=v^if_i\).

\textbf{Definition}: If there exists a basis \(B\subset V\) with
finitely many elements, say, \(d\)-many, then we call \(d\) the
\emph{dimension} of the vector space.

\textbf{Remark}: Let \(V\) be a finite dimensional vector space. Having
chosen a basis \(e_1, ..., e_n\) of \(V\), we may uniquely associate
\begin{equation}
    v \mapsto (v^1, ..., v^n)
\end{equation}

called the \emph{components} of \(v\) with respect to the chosen bases
\(e_1,...,e_n\), where

\begin{equation}
    v = v^1e_1 + ... + v^n e_n
\end{equation}

\hypertarget{basis-for-the-dual-space}{%
\section{Basis for the Dual Space}\label{basis-for-the-dual-space}}

Given an choice of basis \(e_1, ..., e_n\) for \(V\), you can choose a
basis \(\epsilon^1,...\epsilon^n\) for \(V^*\). It is economical to
require that once a basis \(e_j\) has been chosen on \(V\), then
\begin{equation}
    \epsilon^i(e_j) = \delta_j^i
\end{equation} uniquely determines the basis \(\epsilon^i\) for the dual
space.\sidenote{\footnotesize We call such a basis \textbf{the dual basis} of the dual
  space.}

\hypertarget{example-5}{%
\subsection{Example}\label{example-5}}

Consider \(P_3\) with basis \(e_0, e_1, e_2, e_3\) so that \begin{align}
    e_0(x) &= 1 \\ 
    e_1(x) &= x \\ 
    e_2(x) &= x^2 \\ 
    e_3(x) &= x^3
\end{align}

The dual basis \(\epsilon^0, \epsilon^1, \epsilon^2, \epsilon^3\) is
given by \begin{equation}
    \epsilon^a := \frac{1}{a!}\partial^a\Big\vert_{x=0}
\end{equation}

\hypertarget{components-of-tensors}{%
\section{Components of Tensors}\label{components-of-tensors}}

\textbf{Definition}: Let \(T\) be an \((r,s)\)-tensor over a
finite-dimensional vector space \(V\) with basis \(e_1,...e_n\). Then
define the \((r+s)^{\text{dim}(V)}\) many real numbers \begin{equation}
    T^{i_1,...,i_r}_{j_1,...,j_s} = T(\epsilon^{i_1},...\epsilon^{i_r},e_{j_1}, ..., e_{j_s})
\end{equation} where \(\epsilon^i\) is the dual basis for \(V\).

These numbers are called the components of the tensor with respect to
the chosen basis.

This is useful because knowing the components (and the basis from which
they came) allows us to reconstruct the entire tensor.

\hypertarget{example-6}{%
\subsection{Example}\label{example-6}}

Say \(T\) is a \((1,1)\)-tensor. Then it's components are
\begin{equation}
    T^i_j := T(\epsilon^i, e_j)
\end{equation}

To reconstruct T from these components, we have \begin{align}
T(\phi, v) &= T\left(\sum_i \phi_i\epsilon^i, \sum_j v^je_j \right) \\
    & = \sum_i\sum_j \phi_iv^j T(\epsilon^i, e_j) \\ 
    &= \sum_i\sum_j \phi_iv^jT^i_j
\end{align}\sidenote{\footnotesize Going forward, we will use the Einstein sumation
  convention which means \(T(\phi, v) = \phi_iv^jT^i_j\), i.e.~any index
  that is repeated with one up and one down implieas a sum.}

\bookmarksetup{startatroot}

\hypertarget{differentiable-manifolds}{%
\chapter{Differentiable Manifolds}\label{differentiable-manifolds}}

We are now considering differentiable manifolds. So far, we looked at
topological manifolds which, roughly speaking, we would like to extend
to allow us to define a velocity to each point on a curve. Is the
structure \((\mathcal{M}, \mathcal{O})\) enough to talk about
differentiability of curves? No! We need to make more choices (i.e.~add
more structure) before we can discuss differentiablility of a curve.

We wish to define a notion of differentiable\ldots{} \begin{align}
\gamma &:\mathbb{R} \to \mathcal{M} \text{  (curves)  } \\ 
f&:\mathcal{M} \to \mathbb{R} \text{  (functions)  } \\
\phi &: \mathcal{M} \to \mathcal{N} \text{  (maps)  } \\
\end{align}

\hypertarget{strategy}{%
\section{Strategy}\label{strategy}}

Let's consider curves \(\gamma: \mathbb{R} \to \mathcal{M}\). We don't
know what to do with this object on a manifold so what do we do? We
consider the \emph{portion} of the curve that lies in the chart domain
of \((U, x)\).

\%
https://q.uiver.app/?q=WzAsMyxbMCwwLCJcXGdhbW1hOlxcbWF0aGJie1J9Il0sWzIsMCwiXFxtYXRoY2Fse1V9Il0sWzIsMiwieChcXG1hdGhjYWx7VX0pXFxzdWJzZXRlcVxcbWF0aGJie1J9XmQiXSxbMSwyLCJ4Il0sWzAsMV0sWzAsMiwieFxcY2lyY1xcZ2FtbWEiLDJdXQ==

\usepackage{tikz-cd}

{[}

\begin{tikzcd}
    {\gamma:\mathbb{R}} && {\mathcal{U}} \\
    \\
    && {x(\mathcal{U})\subseteq\mathbb{R}^d}
    \arrow["x", from=1-3, to=3-3]
    \arrow[from=1-1, to=1-3]
    \arrow["x\circ\gamma"', from=1-1, to=3-3]
\end{tikzcd}

{]}

\bookmarksetup{startatroot}

\hypertarget{references}{%
\chapter*{References}\label{references}}
\addcontentsline{toc}{chapter}{References}




\end{document}
